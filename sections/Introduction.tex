\clearpage
\section{Introduction}
% ========================================
%     Write your Introduction here
% ========================================

Provide a state-of-the-art and comprehensive literature review in your introduction. Starting with general information, briefly describe an overview of the context of your research background, defining the key concepts, relevant history, and previous work by others. Then narrow down to the specific literature, industry standards, or textbooks that correlate to your work and spotlight the motivation for the project. Highlight the addressed science or engineering problem, the utilized innovation method, and the purpose and usefulness of the device or system you have built. At Last, Summarize the contents of the upcoming chapters as well as the main conclusions of your project, to be elaborated in the last chapter.

All cited references\cite{bibeg} must be listed individually and direct cites from references should be avoided.   Additionally, ensure that your reference is sufficient to support your research and demonstrate how it contributes to your work.

\subsection{Section head}

To create a section head, go to the Styles gallery under the Home tab and pick Heading 2. It automatically formats as above and creates a table of contents entry (after you click the Update tab). Word will not make the capitalization consistent; you have to do that yourself.

It is advisable to set out the general chapter headings in a meaningful way so that reading through them will clearly articulate this narrative. 

Figure \ref{fig:example_fig} is an example of figure and caption style. Table \ref{tab:example_table} is an example of table and table title style. A starter table for parts costs is in Chapter 4 of this template.

Use the References Insert Caption tool to generate consistently formatted captions (always below the figure), and use the grouping function in Word’s drawing tools to hold figure and caption together. Use picture formatting tools to hold figures in place (preferably at top or bottom of page) and to define text wraps (“top and bottom” is best).

Use Word's table design and layout tools to format titles, column heads, and borders.

Insert page break at end of every chapter to ensure next chapter starts on new page.

\begin{figure}[H]
    \centering
    \includegraphics[width=\textwidth]{figures/block.png}
    \caption{Example of placement and caption for a block diagram.}
    \label{fig:example_fig}
\end{figure}

\begin{table}[H]
    \centering
    \caption{Example of a Table and Its Title}
    \label{tab:example_table}
    \begin{tabular}{ccc}
        \hline
        \textbf{Part} & \textbf{Electricity} & \textbf{Magnetism} \\ \hline
        Field intensity & $E$ & $H$ \\
        Flux density & $D$ & $B$ \\
        Constitutive factor & $\varepsilon^b$ & $\mu^c$ \\\hline
    \end{tabular}
\end{table}
